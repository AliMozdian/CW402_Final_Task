\documentclass{article}
\title{Computer Workshop \\ Final Assignment}
\author{Ali Mozdianfard}
\date{January 2024}

\usepackage{graphicx}

\usepackage{listings}
\usepackage{color}


\definecolor{dkgreen}{rgb}{0,0.6,0}
\definecolor{gray}{rgb}{0.5,0.5,0.5}
\definecolor{mauve}{rgb}{0.58,0,0.82}

\lstset{frame=tb,
	language={},
	aboveskip=3mm,
	belowskip=3mm,
	showstringspaces=false,
	columns=flexible,
	basicstyle={\small\ttfamily},
	numberstyle=\tiny\color{gray},
	keywordstyle=\color{blue},
	commentstyle=\color{dkgreen},
	stringstyle=\color{mauve},
	breaklines=true,
	breakatwhitespace=true,
	tabsize=4,
	numbers=left,
	stepnumber=1
}

\begin{document}
	\maketitle
	\newpage
	
	\tableofcontents
	\newpage
	
	\section{Git \& GitHub}
	Here is every step of the process of creating the repository of this assignment
	and setting up the GitHub Actions for compiling \LaTeX.
	
	\begin{figure}[h]
		\centering
		\includegraphics[width=\textwidth]{images/ss0.png}
		\caption{Creating the repository in GitHub}
	\end{figure}
	
	\begin{figure}[h]
		\centering
		\includegraphics[width=\textwidth]{images/ss1.png}
		\caption{Copying the link of HTTPS link for cloning the repository into my local machine}
	\end{figure}
	
	\begin{figure}[h]
		\centering
		\includegraphics[width=\textwidth]{images/ss2.png}
		\caption{From now on we will be doing the needed setup for GitHub Actions. \\ Creating the needed files for the project.}
	\end{figure}
	
	\begin{figure}[h]
		\centering
		\includegraphics[width=\textwidth]{images/ss3.png}
		\caption{Writing in main.yml to set GitHub Actions (.github/workflows/main.yml)}
	\end{figure}
	
	\begin{figure}[h]
		\centering
		\includegraphics[width=\textwidth]{images/ss4.png}
		\caption{add, commit and push, then we are ready to start the project, but tags!}
	\end{figure}
	
	\begin{figure}[h]
		\centering
		\includegraphics[width=\textwidth]{images/ss5.png}
		\caption{Adding and commiting the images, also making a tag for this commit. \\ then pushing the repo and its tags (via two commands)}
	\end{figure}
	
	\begin{figure}[h]
		\centering
		\includegraphics[width=\textwidth]{images/ss6.png}
		\caption{As you can see the tags now are in GitHub as well}
	\end{figure}
	
	\begin{figure}[h]
		\centering
		\includegraphics[width=\textwidth]{images/ss7.png}
		\caption{And becuase of the format of the tag, GitHub Actions complied it without a problem}
	\end{figure}
	
	So just repeat this cycle:
	\begin{enumerate}
		\item make changes
		\item git add
		\item git commit
		\item git tag
		\item git push
		\item git push tags
		\item GitHub Actions compiles the main.tex automatically
	\end{enumerate}

	\section{Exploration Tasks}
	\subsection{Vim Advanced Features}
	In this section we are going to learn about three advanced features of VIM that you might doesn't expect them:
	\begin{enumerate}
		\item Multiple Windows and Tabs
		\item Folding
		\item Scripting
	\end{enumerate}

	Here I've search about them and made a short document for each of these 3 above.

	\subsubsection{Multiple Windows and Tabs}
	\paragraph{Multiple Windows:} In Vim, windows refer to individual panes within the editor's interface, each displaying a different section of the same or different file.
	By splitting the editing window into multiple windows, users can view and edit different parts of a file concurrently, facilitating seamless multitasking and improving workflow efficiency.

	Some of the key commands to do so:
	\begin{itemize}
		\item ":vsp" -> splits window vertically (or you can use ":vertical split")
		\item ":sp" -> splits window horizontally (or you can use ":split")
		\item "Ctrl + w" -> navigate between windows (use it followed by arrow keys to do it more efficiently)
		\item "Ctrl + w" followed by '>' or '<'' to increase or decrease window size.
	\end{itemize}

	\paragraph{Tabs:} Vim also supports the use of tabs, allowing users to work with multiple files within the same Vim session.
	Each tab represents a separate editing session, with its own set of windows and buffers, providing a convenient way to organize and
	switch between different files or projects.

	Some of the key commands to do so:
	\begin{itemize}
		\item ":tabnew" or ":tabnew <filename>" -> opens new tab
		\item "gt" -> switch to the next tab
		\item "gT" -> switch to the previous tab
		\item ":tabclose" -> tabclose
	\end{itemize}

	\subsubsection{Folding}
	Folding in Vim is a feature that allows you to collapse or "fold" sections of text within a file, making it easier to focus on specific parts of the document while hiding others.
	It's particularly useful for navigating large files, organizing code, and managing complex structures such as functions, classes, or sections of documentation.
	In general, there is two types of folding in VIM, Manual Folding and indentation-based Folding.

	Some of the key commands to do so:
	\begin{itemize}
		\item zf followed by a motion -> create a fold for the text covered by the specified motion.
		\item za -> toggle fold and unfold at the position that cursor points at.
		\item zd -> delete the fold at the cursor without deleting its contents, just the fold itself.
	\end{itemize}

	\subsubsection{Scripting}
	Vimscript is a powerful scripting language built into Vim, allowing users to automate tasks and customize Vim's behavior extensively.
	Vimscript functions can be defined to perform specific actions, and custom commands can be created to execute sequences of Vim commands conveniently
	Example of defining a Vimscript function to perform a custom action:

	\begin{lstlisting}
	function! MyCustomFunction()
    	" Vimscript code goes here
		endfunction
	\end{lstlisting}

	Example of creating a custom command to execute the above function:

	\begin{lstlisting}
	command! MyCommand call MyCustomFunction()
	\end{lstlisting}


	With custom commands and functions, users can automate repetitive tasks, create complex editing workflows, and enhance Vim's functionality to suit their specific needs.



\end{document}
